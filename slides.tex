\documentclass{beamer}
\title{Guarded recursion in the topos of trees}
\author{Bálint Kocsis}
\institute{Radboud University}
\date{\today}

\usetheme{Boadilla}
\usepackage[utf8]{inputenc}
\usepackage{microtype}
\usepackage{xparse}

\usepackage{amsmath}
\usepackage{amssymb}
\usepackage{amsthm}
\usepackage{mathpartir}
\usepackage{stmaryrd}
\usepackage{tikz-cd}

\usepackage[hidelinks,colorlinks,linkcolor=magenta,citecolor=olive,urlcolor=cyan]{hyperref}
\usepackage[capitalise]{cleveref}

\usepackage{colonequals}
\usepackage{minted}

\usepackage{pftools}

\crefname{defn}{Definition}{Definitions}
\crefname{ex}{Example}{Examples}
\crefname{equation}{Equation}{Equations}
\crefname{figure}{Figure}{Figures}

\DeclareUnicodeCharacter{00D7}{$\times$}
\DeclareUnicodeCharacter{0393}{$\Gamma$}
\DeclareUnicodeCharacter{03A9}{$\Omega$}
\DeclareUnicodeCharacter{03B3}{$\gamma$}
\DeclareUnicodeCharacter{03C0}{$\pi$}
\DeclareUnicodeCharacter{03BB}{$\lambda$}
\DeclareUnicodeCharacter{2016}{$\|$}
\DeclareUnicodeCharacter{2081}{$_1$}
\DeclareUnicodeCharacter{2192}{$\to$}
\DeclareUnicodeCharacter{21D2}{$\To$}
\DeclareUnicodeCharacter{2200}{$\forall$}
\DeclareUnicodeCharacter{2218}{$\circ$}
\DeclareUnicodeCharacter{2264}{$\le$}
\DeclareUnicodeCharacter{227A}{$\prec$}
\DeclareUnicodeCharacter{22A2}{$\vdash$}
\DeclareUnicodeCharacter{25B7}{$\later$}
\DeclareUnicodeCharacter{27E6}{$\llbracket$}
\DeclareUnicodeCharacter{27E7}{$\rrbracket$}
\DeclareUnicodeCharacter{27E8}{$\langle$}
\DeclareUnicodeCharacter{27E9}{$\rangle$}
\DeclareUnicodeCharacter{27F6}{$\longrightarrow$}
\DeclareUnicodeCharacter{2AAF}{$\preceq$}

\definecolor{rocqkw}{rgb}{0,0.5,0}

\parindent = 0.7\parindent

\theoremstyle{definition}
\newtheorem{defn}{Definition}[subsection]
\newtheorem{ex}[defn]{Example}

\theoremstyle{plain}
\newtheorem{prop}[defn]{Proposition}

\theoremstyle{remark}
\newtheorem{rem}[defn]{Remark}

\newminted[rocqc]{coq}{escapeinside = //, samepage}
\newmintinline[rocqi]{coq}{escapeinside = //}

\newcommand{\ol}{\overline}
\newcommand{\To}{\Rightarrow}
\renewcommand{\phi}{\varphi}
\newcommand{\subs}{\subseteq}
\newcommand{\nul}{\emptyset}
\renewcommand{\iff}{\text{ iff }}

\newcommand{\obj}{\mathtt}
\newcommand{\mor}{\mathtt}
\newcommand{\name}{\mathtt}
\newcommand{\sep}{\mathbin{|}}
\newcommand{\setof}[2]{\{ #1 \sep #2 \}}

\newcommand{\N}{\mathbb{N}}
\newcommand{\nsuc}{\mor{succ}}
\newcommand{\2}{\mathbf{2}}
\newcommand{\btrue}{\mor{true}}
\newcommand{\bfalse}{\mor{false}}
\newcommand{\fin}[1]{{[0..#1]}}
\newcommand{\fino}[1]{{[0..#1)}}

\newcommand{\cat}[1]{\mathcal{#1}}
\newcommand{\Hom}[3][]{\mathrm{Hom}_{#1}(#2, #3)}
\newcommand{\id}[1][]{\mor{id}_{#1}}
\newcommand{\0}{\mathbf{0}}
\newcommand{\1}{\mathbf{1}}
\newcommand{\minit}{{!}}
\newcommand{\mterm}{{!}}
\newcommand{\mprod}[2]{(#1, #2)}
\newcommand{\mprojl}{\pi_1}
\newcommand{\mprojr}{\pi_2}
\newcommand{\msum}[2]{[#1, #2]}
\newcommand{\minl}{\kappa_1}
\newcommand{\minr}{\kappa_2}
\newcommand{\expt}[2]{#1 \To #2}
\newcommand{\ev}{\mor{ev}}
\newcommand{\curry}[1]{\lambda #1}
\newcommand{\transpose}[1]{\ulcorner #1 \urcorner}
\newcommand{\fix}[1]{\mu #1}
\newcommand{\SOC}{\Omega}
\newcommand{\pow}[1]{\expt{#1}{\SOC}}

\newcommand{\Set}{\mathbf{Set}}
\newcommand{\TOT}{\cat{S}}

\newcommand{\restr}[2]{#1|_{#2}}
\newcommand{\oStr}{\obj{Str}}
\newcommand{\mhd}{\mor{hd}}
\newcommand{\mtl}{\mor{tl}}
\newcommand{\mcons}{\mor{cons}}
\DeclareMathOperator{\Later}{\blacktriangleright\hspace{-1pt}}
\newcommand{\mnxt}{\mor{next}}
\newcommand{\mtrue}{\mor{true}}
\newcommand{\mfalse}{\mor{false}}
\newcommand{\mconj}{\mor{conj}}
\newcommand{\mdisj}{\mor{disj}}
\newcommand{\mimpl}{\mor{impl}}
\newcommand{\mall}{\mor{all}}
\newcommand{\mex}{\mor{ex}}
\newcommand{\mlater}{\triangleright}
\newcommand{\mlift}{\mor{lift}}
\newcommand{\meq}{\mor{eq}}

\newcommand{\lang}{\mathcal{L}}
\newcommand{\unitt}{\1}
\newcommand{\emptyt}{\0}
\newcommand{\Prop}{\name{Prop}}
\newcommand{\unit}{()}
\newcommand{\pair}[2]{(#1, #2)}
\newcommand{\projl}[1]{\pi_1 #1}
\newcommand{\projr}[1]{\pi_2 #1}
\newcommand{\abort}[1]{\name{abort}\, #1}
\newcommand{\inl}[1]{\name{inj}_1\, #1}
\newcommand{\inr}[1]{\name{inj}_2\, #1}
\newcommand{\case}[5]{\name{case}\, #1\, \name{of}\, #2. #3; #4. #5}
\newcommand{\lam}[3]{\lambda #1 : #2.\,#3}
\newcommand{\app}[2]{#1\, #2}
\newcommand{\nxt}[1]{\name{next}\, #1}
\newcommand{\ap}{\circledast}
\newcommand{\rec}[3]{\mu #1 : #2.\,#3}
\newcommand{\true}{\top}
\newcommand{\false}{\bot}
\newcommand{\impl}{\supset}
\newcommand{\all}[3]{\forall #1 : #2.\,#3}
\newcommand{\exi}[3]{\exists #1 : #2.\,#3}
\newcommand{\later}{\triangleright}
\newcommand{\lift}[1]{\name{lift}\, #1}

\newcommand{\subst}[3]{#1\left[#2 := #3\right]}
\newcommand{\conext}[3]{#1, #2 : #3}
\newcommand{\typed}[3]{#1 \vdash #2 : #3}
\newcommand{\entails}[3]{#1 \sep #2 \vdash #3}
\newcommand{\valid}[2]{#1 \vdash #2}
\newcommand{\dashvdash}{\mathbin{\dashv\vdash}}
\renewcommand{\equiv}[3]{#1 \sep #2 \dashvdash #3}
\newcommand{\equals}[4][]{#2 \vdash #3 =_{#1} #4}

\newcommand{\Str}{\name{Str}}
\newcommand{\suc}{\name{succ}}
\newcommand{\hd}{\name{hd}}
\newcommand{\tl}{\name{tl}}
\newcommand{\cons}{\coloncolon}

\newcommand{\sem}[1]{\llbracket #1 \rrbracket}
\newcommand{\sementails}[3]{#1 \sep #2 \vDash #3}
\newcommand{\force}[2]{#1 \Vdash #2}

\newcommand{\lamb}[2]{\lambda #1. #2}
\newcommand{\recu}[2]{\mu #1. #2}

\begin{document}

\maketitle

\begin{frame}{Step-indexing}
\begin{itemize}
    \item<1-> Problem: non-well-founded definitions, e.g.
    \[ X = X \To A \]
    \item<2-> Solution: stratify recursion using sequences of successive approximations
    \begin{align*}
        X_{-1} &= V \\
        X_{n+1} &= X_n \cap (X_n \To A) \\
        X &= \bigcap{X_n} \\
    \end{align*}
    \item<3-> \vspace{-20pt}
    Syntactically: introduce modality $\Later$ to talk about the next step
    \[ T = \mu X. \Later{X} \to A \]
\end{itemize}
\end{frame}

\begin{frame}{Applications}
\begin{itemize}
    \item Solve recursive domain equations
    \item Model general recursive types
    \item Model general references
    \item Define recursive functions, including negative self-references
    \item Concrete projects
    \begin{itemize}
        \item Typed intermediate/assembly languages: FPCC
        \item Program logics: VST, Iris
        \item Guarded type theory
    \end{itemize}
\end{itemize}
\end{frame}

\begin{frame}{Focus of my work}
\begin{itemize}
    \item Logic of step-indexing/guarded recursion
    \item Particular model: topos of trees
    \item Formalization in Coq
\end{itemize}
\end{frame}

\begin{frame}{Guarded type theory}
\begin{itemize}
    \item<1-> New type former $\Later$
    \begin{itemize}
        \item Allows us to talk about data we will only have access to in the next computation step
        \item Guards self-references in type/term definitions:
        \[ T = \mu X. \Later{X} \to A \]
    \end{itemize}
    % The type $\Later{A}$ expresses that we will only have access to the underlying value of type $A$ one step later
    \item<2-> Applicative structure
    \begin{itemize}
        \item $\name{next} : A \to \Later{A}$
        \item $- \ap - : \Later{(A \to B)} \to \Later{A} \to \Later{B}$
    \end{itemize}
    \item<3-> Guarded fixed point combinator: $\name{fix} : (\Later{A} \to A) \to A$
    \begin{itemize}
        \item Self-reference delayed in time by $\nxt$:
        \[ \app{\name{fix}}{f} = \app{f}{(\nxt{(\app{\name{fix}}{f})})} \]
        \item We write $\rec{x}{\Later{A}}{t}$ for
        $\app{\name{fix}}{\lam{x}{\Later{A}}{t}}$
    \end{itemize}
\end{itemize}
\end{frame}

\begin{frame}{Motivating example: streams}
\begin{itemize}
    \item<1-> $\Str = \mu X. \N \times \Later{X}$, hence $\Str = \N \times \Later{\Str}$
    \item<2-> Constructors and destructors: \vspace{-6pt}
    \begin{align*}
        - \cons - &: \N \to \Later{\Str} \to \Str \\
        \name{hd} &: \Str \to \N \\
        \name{tl} &: \Str \to \Later{\Str} \\
    \end{align*}
    \item<3-> Recursive operations: \vspace{-6pt}
    \begin{align*}
        \name{zeros} &= \recu{s}{0 \cons s} : \Str \\
        \name{add} &= \lamb{n}{\recu{r}{\lamb{s}{
            (\app{\hd}{s} + n) \cons (r \ap \app{\tl}{s})}}} : \N \to \Str \to \Str \\
    \end{align*}
\end{itemize}
\end{frame}

\begin{frame}{Step-indexed logic}
Under the Curry-Howard correspondence, guarded recursive operations correspond to logical connectives and rules.
\begin{itemize}
    \item<2-> $\Later$ type former $\To$ $\later$ modality
    \begin{itemize}
        \item $\later{P}$ holds if $P$ holds at the next step
    \end{itemize}
    \item<3-> Applicative structure $\To$ modal axioms:
        \begin{mathpar}
            \inferrule{}{P \vdash \later{P}}
            \and
            \inferrule{}{\later{(P \impl Q)} \wedge \later{P} \vdash \later{Q}}
        \end{mathpar}
    \item<4-> Fixed point combinator $\To$ Löb rule:
    \begin{mathpar}
        \inferrule{}{\later{P} \impl P \vdash P}
    \end{mathpar}
    In short: to prove $P$, we can assume that $P$ already holds after one computation step
\end{itemize}
\end{frame}

\begin{frame}{Some important rules}
\begin{mathpar}
    \inferrule{}{P \vdash \later{P}}
    \and
    \inferrule{P \vdash Q}{\later{P} \vdash \later{Q}}
    \and
    \inferrule{\later{P} \vdash P}{\vdash P}
    \and
    \inferrule{}{\later{P} \impl P \vdash P}
    \\
    \inferrule{}{\later{(P \ast Q)} \dashvdash \later{P} \ast \later{Q}}
        \quad (\ast \in \{\wedge, \vee, \impl\})
    \and
    \inferrule{}{\later{(t =_A u)} \dashvdash \nxt{t} =_{\Later{A}} \nxt{u}}
\end{mathpar}
\end{frame}

\begin{frame}{Semantics}
\begin{itemize}
    \item Intuitively: sequences of approximations
    \begin{itemize}
         \item The $n$-th element describes what the object looks like if one has only $n$ steps to reason about it
        \item $n$: step-index
         \item $\Later$ and $\later$ shift step-indices
    \end{itemize}
    \item Two main formalisms:
    \begin{itemize}
        \item Ordered families of equivalences (used by Iris): a set equipped with more and more refined equivalence relations
        \item Topos of trees: sequence of sets with restriction maps
    \end{itemize}
\end{itemize}
\end{frame}

\begin{frame}[fragile]{Topos of trees}
\begin{itemize}
    \item $\TOT$: presheaves on the ordinal $\omega$
    \item Objects $X$:
    % https://q.uiver.app/#q=WzAsNCxbMCwwLCJYXzAiXSxbMSwwLCJYXzEiXSxbMiwwLCJYXzIiXSxbMywwLCJcXGNkb3RzIl0sWzEsMCwicl5YXzAiLDJdLFsyLDEsInJeWF8xIiwyXSxbMywyLCJyXlhfMiIsMl1d
    \[\begin{tikzcd}
    	{X_0} & {X_1} & {X_2} & \cdots
    	\arrow["{r^X_0}"', from=1-2, to=1-1]
    	\arrow["{r^X_1}"', from=1-3, to=1-2]
    	\arrow["{r^X_2}"', from=1-4, to=1-3]
    \end{tikzcd}\]
    \item Morphisms $f : X \to Y$:
    % https://q.uiver.app/#q=WzAsOCxbMCwwLCJYXzAiXSxbMSwwLCJYXzEiXSxbMiwwLCJYXzIiXSxbMywwLCJcXGNkb3RzIl0sWzAsMSwiWV8wIl0sWzEsMSwiWV8xIl0sWzIsMSwiWV8yIl0sWzMsMSwiXFxjZG90cyJdLFsxLDAsInJeWF8wIiwyXSxbMiwxLCJyXlhfMSIsMl0sWzMsMiwicl5YXzIiLDJdLFs1LDQsInJeWV8wIiwyXSxbNiw1LCJyXllfMSIsMl0sWzcsNiwicl5ZXzIiLDJdLFswLDQsImZfMCIsMl0sWzEsNSwiZl8xIiwyXSxbMiw2LCJmXzIiLDJdXQ==
    \[\begin{tikzcd}
    	{X_0} & {X_1} & {X_2} & \cdots \\
    	{Y_0} & {Y_1} & {Y_2} & \cdots
    	\arrow["{r^X_0}"', from=1-2, to=1-1]
    	\arrow["{r^X_1}"', from=1-3, to=1-2]
    	\arrow["{r^X_2}"', from=1-4, to=1-3]
    	\arrow["{r^Y_0}"', from=2-2, to=2-1]
    	\arrow["{r^Y_1}"', from=2-3, to=2-2]
    	\arrow["{r^Y_2}"', from=2-4, to=2-3]
    	\arrow["{f_0}"', from=1-1, to=2-1]
    	\arrow["{f_1}"', from=1-2, to=2-2]
    	\arrow["{f_2}"', from=1-3, to=2-3]
    \end{tikzcd}\]
    %\item intuition?
    %\item necessary constructions (e.g. products, exponentials)
\end{itemize}
\end{frame}

\begin{frame}[fragile]{Guarded recursion}
\begin{itemize}
    \item $\Later : \TOT \to \TOT$ sends $X$ to
    % https://q.uiver.app/#q=WzAsNCxbMCwwLCJcXHsqXFx9Il0sWzEsMCwiWF8wIl0sWzIsMCwiWF8xIl0sWzMsMCwiXFxjZG90cyJdLFsxLDAsIiEiLDJdLFsyLDEsInJfMCIsMl0sWzMsMiwicl8xIiwyXV0=
    \[\begin{tikzcd}
    	{\{*\}} & {X_0} & {X_1} & \cdots
    	\arrow["{!}"', from=1-2, to=1-1]
    	\arrow["{r_0}"', from=1-3, to=1-2]
    	\arrow["{r_1}"', from=1-4, to=1-3]
    \end{tikzcd}\]
    \item $\mnxt_X : X \to \Later{X}$, $(\mnxt_X)_n = r^{\Later{X}}_n$
    % https://q.uiver.app/#q=WzAsOCxbMCwwLCJYXzAiXSxbMSwwLCJYXzEiXSxbMiwwLCJYXzIiXSxbMCwxLCJcXHsqXFx9Il0sWzEsMSwiWF8wIl0sWzIsMSwiWF8xIl0sWzMsMCwiXFxjZG90cyJdLFszLDEsIlxcY2RvdHMiXSxbMSwwLCJyXlhfMCIsMl0sWzIsMSwicl5YXzEiLDJdLFs2LDIsInJeWF8yIiwyXSxbNSw0LCJyXlhfMCJdLFs3LDUsInJeWF8xIl0sWzAsMywiISIsMl0sWzQsMywiISJdLFsxLDQsInJeWF8wIiwyXSxbMiw1LCJyXlhfMSIsMl1d
    \[\begin{tikzcd}
    	{X_0} & {X_1} & {X_2} & \cdots \\
    	{\{*\}} & {X_0} & {X_1} & \cdots
    	\arrow["{r^X_0}"', from=1-2, to=1-1]
    	\arrow["{r^X_1}"', from=1-3, to=1-2]
    	\arrow["{r^X_2}"', from=1-4, to=1-3]
    	\arrow["{r^X_0}", from=2-3, to=2-2]
    	\arrow["{r^X_1}", from=2-4, to=2-3]
    	\arrow["{!}"', from=1-1, to=2-1]
    	\arrow["{!}", from=2-2, to=2-1]
    	\arrow["{r^X_0}"', from=1-2, to=2-2]
    	\arrow["{r^X_1}"', from=1-3, to=2-3]
    \end{tikzcd}\]
\end{itemize}
\end{frame}

\begin{frame}[fragile]{Examples}
\begin{itemize}
    \item Streams:
    % https://q.uiver.app/#q=WzAsNCxbMCwwLCJcXE4iXSxbMSwwLCJcXE4gXFx0aW1lcyBcXE4iXSxbMiwwLCJcXE4gXFx0aW1lcyBcXE4gXFx0aW1lcyBcXE4iXSxbMywwLCJcXGNkb3RzIl0sWzEsMCwiXFxwaV8xIiwyXSxbMiwxXSxbMiwxLCJcXHBpXzEiLDJdLFszLDIsIlxccGlfMSIsMl1d
    \[\begin{tikzcd}
    	\N & {\N \times \N} & {\N \times \N \times \N} & \cdots
    	\arrow["{\pi_1}"', from=1-2, to=1-1]
    	\arrow[from=1-3, to=1-2]
    	\arrow["{\pi_1}"', from=1-3, to=1-2]
    	\arrow["{\pi_1}"', from=1-4, to=1-3]
    \end{tikzcd}\]
    \item $\mhd_n(s_0, \ldots, s_n) = s_0$
    \item $\name{inc}_n(s_0, \ldots, s_n) = (s_0 + 1, \ldots, s_n + 1)$
\end{itemize}
\end{frame}

\begin{frame}{Guarded fixed points}
\begin{prop}
Let $f : \Later{X} \to X$ be a morphism of $\TOT$. Then there exists a unique
$x : \1 \to X$ such that $f \circ \mnxt_X \circ x = x$.
\end{prop}
\begin{proof}
We have $f_0 : \{*\} \to X_0$ and $f_{n+1} : X_n \to X_{n+1}$.
Define $x : \1 \to X$ by recursion: $x_0 = f_0$ and $x_{n+1} = f_{n+1} \circ x_n$.
\end{proof}
\end{frame}

\begin{frame}{Logic}
\begin{itemize}
    \item Essentially Kripke semantics over the natural numbers
    \item Intuition: the truth of a proposition depends on the step-index $n$, i.e. the amount of computation steps left
    \item If $P$ is true for $n$ steps, then it is also true for less than $n$ steps
    \item Hence: a truth value is a downward closed subset of step indices
\end{itemize}
\end{frame}

\begin{frame}{Kripke-Joyal semantics}
Forcing relation: $\force{n}{P}$ iff $P$ holds at step $n$
\begin{prop} \label{prop:kripke-joyal}
    The forcing relation satisfies the following clauses:
    \begin{align*}
        \force{n}{P \impl Q} &\iff
          \forall m \le n: \force{m}{P} \To \force{m}{Q} \\
        \force{n}{\exi{x}{A}{P}} &\iff
          \exists a \in \sem{A}_n: \force{n}{P(a)} \\
        \force{n}{\all{x}{A}{P}} &\iff
          \forall m \le n, a \in \sem{A}_m: \force{m}{P(a)} \\
        \force{n}{\later{P}} &\iff
          n = 0 \vee \force{n-1}{P}
    \end{align*}
\end{prop}
\end{frame}

\begin{frame}{$\later$ and quantifiers}
\begin{itemize}
    \item We have
    \begin{mathpar}
        \inferrule{}{\exi{x}{A}{\later{P}} \vdash \later{(\exi{x}{A}{P})}}
        \and
        \inferrule{}{\later{(\all{x}{A}{P})} \vdash \all{x}{A}{\later{P}}}
    \end{mathpar}
    \item However, the other directions are not valid, e.g.
    \begin{align*}
        \force{n+1}{\later{(\exi{x}{A}{P})}} &\iff
            \exists a \in \sem{A}_n: \force{n}{P(a)} \\
        \force{n+1}{\exi{x}{A}{\later{P}}} &\iff
            \exists a \in \sem{A}_{n+1}: \force{n}{P(\restr{a}{n})} \\
    \end{align*}
    \item There does not seem to be a general rule for commuting $\later$ with a quantifier
\end{itemize}
\end{frame}

\begin{frame}{$\mlift$}
\begin{itemize}
    \item We can decompose $\later = \mlift \circ \mnxt$ \cite{clouston:2017:lmcs}
    \item Hence, we could investigate the properties of $\mlift$
    \item Novel rule:
    \begin{mathpar}
        \inferrule{}
            {\lift{(\nxt{\mex} \ap Q)} \dashvdash
             \exi{y}{\Later{A}}{\lift{(Q \ap y)}}}
    \end{mathpar}
    where
    \[ Q : \Later{(A \to \Prop)} \]
    \[ \mex = \lam{P}{A \to \Prop}{\exi{x}{A}{\app{P}{x}}} \]
\end{itemize}
\end{frame}

\begin{frame}[fragile]{Coq formalization}
\begin{itemize}
    \item Need finite types for the definition of propositions
    \item Usual representation:
\begin{coqc}
Inductive fin : nat → Type :=
  | FZ {n} : fin (S n)
  | FS {n} : fin n → fin (S n).
\end{coqc}
    \item Alternatively:
\begin{coqc}
Definition fin (n : nat) := {m : nat | m ≺ n}.
\end{coqc}
    \item The latter definition works much better in practice
\end{itemize}
\end{frame}

\begin{frame}{Conclusion}
\begin{itemize}
    \item Exposition of the topos of trees
    \item Emphasis on $\mlift$, which seems to be more fundamental
    \item Coq formalization: case study in using proof-irrelevant propositions for the representation of finite types
\end{itemize}
\end{frame}

\begin{frame}{Future work}
\begin{itemize}
    \item Find appropriate rules for $\mlift$
    \item Investigate step-indexed logic from the perspective of modal type theory
    \item Formalize a model of Iris in guarded type theory
\end{itemize}
\end{frame}

\begin{frame}{References}
\bibliographystyle{abbrv}
\bibliography{refs}
\end{frame}

\end{document}